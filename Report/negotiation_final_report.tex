\documentclass[a4paper,10pt]{article}
%\documentclass[a4paper,10pt]{scrartcl}

\usepackage[utf8]{inputenc}
\usepackage{amsmath}
\usepackage{listings}
\usepackage{hyperref}
\usepackage{catoptions}
\usepackage[margin=1in]{geometry}
\usepackage{color}
\usepackage{soul}
\usepackage{float}
\usepackage{framed}
\usepackage[sc]{mathpazo}
\linespread{1.20}         % Palatino needs more leading (space between lines)
\usepackage[T1]{fontenc}
\usepackage{microtype}
\usepackage{enumerate}
\usepackage{courier}
\usepackage{graphicx}
\usepackage{xfrac}
\usepackage{enumitem}

\newcommand{\authorbd}{Bas Dado (4033736)}
\newcommand{\authoroh}{Olivier Hokke (1352679)}
\newcommand{\authortr}{Tom Runia (1517996)}
\newcommand{\authoras}{Arnold Schutter (4260724)}
\newcommand{\authortv}{Tom Viering (4333055)}
\newcommand{\maintitle}{Artificial Intelligence}
\newcommand{\subtitle}{BOAconstructor -- An Automated Negotiation Agent}

\title{\maintitle\\\subtitle}
\author{\authorbd\\\authoroh\\\authortr\\\authoras\\\authortv}
\date{\today}

\pdfinfo{%
  /Title    (\maintitle - \subtitle)
  /Author   (\authorbd, \authoroh, \authortr, \authoras, \authortv)
  /Creator  (\authorbd, \authoroh, \authortr, \authoras, \authortv)
  /Producer (\authorbd, \authoroh, \authortr, \authoras, \authortv)
  /Subject  (Automated Negotiation)
  /Keywords (Automated Negotiation, Genius, Bidding Strategy, Acceptance Strategy, Opponent Model, Opponent Model Strategy)
}

% Settings for hyperref package (e.g. wat \autoref en \nameref moeten doen)
\hypersetup{
  bookmarks   = false
  colorlinks  = false,
  linkcolor   = [rgb]{0.1,0.1,0.5},
  citecolor   = [rgb]{0.5,0.1,0.1},
  filecolor   = [rgb]{0.1,0.5,0.5},
  urlcolor    = [rgb]{0.1,0.1,0.7}
}

% Adds the command "\Autoref" to make it possible to use a capital in the referenced object name
\makeatletter
\def\figureautorefname{figure}
\def\tableautorefname{table}
\def\Autoref#1{%
  \begingroup
  \edef\reserved@a{\cpttrimspaces{#1}}%
  \ifcsndefTF{r@#1}{%
    \xaftercsname{\expandafter\testreftype\@fourthoffive}
      {r@\reserved@a}.\\{#1}%
  }{%
    \ref{#1}%
  }%
  \endgroup
}
\def\testreftype#1.#2\\#3{%
  \ifcsndefTF{#1autorefname}{%
    \def\reserved@a##1##2\@nil{%
      \uppercase{\def\ref@name{##1}}%
      \csn@edef{#1autorefname}{\ref@name##2}%
      \autoref{#3}%
    }%
    \reserved@a#1\@nil
  }{%
    \autoref{#3}%
  }%
}
\makeatother

% Settings for listings of java code
\definecolor{mygreen}{rgb}{0,0.6,0}
\definecolor{light-gray}{gray}{0.95}
\lstset{basicstyle=\footnotesize\ttfamily,breaklines=true,language=Java}
\lstset{frame=single,commentstyle=\color{mygreen},keywordstyle=\color{blue}}
\lstset{aboveskip=0.5cm,belowskip=0.3cm}
\lstset{backgroundcolor=\color{light-gray}}

% Define the todo command
\newcommand{\todo}[1] {\hl{TODO: #1}}
\setlength{\parindent}{0cm}

\begin{document}

\begin{center}
\vskip 1cm
{\Huge Artificial Intelligence \vskip 2mm}
{\Large Automated Negotation: \\BOA constructor agent\vskip 1cm}

{\normalsize \textbf{Bas Dado} ($4033736$) -- \textbf{Olivier Hokke} ($1352679$) -- \textbf{Tom Runia} ($1517996$) \vskip 2mm \textbf{Tom Viering} ($4333055$) -- \textbf{Arnold Schutter} ($4260724$) \vskip 3cm}

\end{center}

\begin{abstract}
\noindent A relatively new and evolving branch of artificial intelligence is \emph{automated negotiation}. In automated negotiation, two or more agents negotiate about a multi-issue problem in order to find a solution that maximizes their utility. The utilities for each possible bid are determined using a human-defined preference profile, which consists of weights for each of the issues and a utility for each possible value of the issues. \\

\noindent In this report we describe the process of creating our agent, \texttt{BOAconstructor}, for automated negotiation. \Autoref{sec:exercises} contains the answers to the questions posed in the assignment concerning the party domain and genius in general. 
\Autoref{sec:strategy} describes the strategy our agent uses. The main chapter contains the high-level description of the agent. In the subsections, \autoref{sec:strategyAS}, \autoref{sec:strategyBS}, \autoref{sec:strategyOM} and \autoref{sec:strategyOMS}, we go into more detail about the specific \emph{BOA components}. \Autoref{sec:performance} shows the results of performance tests of our final agent against a number agents that were included in genius. In \Autoref{sec:conclusion} we describe our experience regarding building the agent and decide what is needed in order to use our agent in real world negotiations.
\end{abstract}

\newpage
\tableofcontents
\newpage


\section{Assignments}
\label{sec:exercises}

\subsection{Analysis of the Party domain}

\subsubsection{Bidding Space and Pareto Frontier}

In this subsection we analyse the \emph{Party domain} and compute the Pareto Frontier. The party domain is a negotiation domain which consists of $6$ issueas that each contain three or four possibilities: 

\begin{enumerate}[itemsep=0mm]
  \item \textbf{Food} (4 options)
  \item \textbf{Drinks} (4 options)
  \item \textbf{Location} (4 options)
  \item \textbf{Invitations} (4 options)
  \item \textbf{Music} (3 options)
  \item \textbf{Clean-Up} (4 options)
\end{enumerate}

The issues and possible options are good for a total of $3072$ possible outcomes ($4 \times 4 \times 4 \times 4 \times 3 \times 4$). We have used the \emph{preference profiles} of users $10$ and $15$ and analyzed these to compute the \emph{outcome space} and \emph{Pareto Frontier}. Both of these are given in the figure below.

\begin{figure}[H]
\begin{center}
 \includegraphics[width=0.8\textwidth]{pareto.png}
 \caption{The outcome space with the \emph{Pareto Optimal Frontier} indicated as a red line.}
 \label{fig:pareto} 
\end{center}
\end{figure}

\clearpage

\subsubsection{Performance of the \texttt{SimpleAgent}}

Finally, we perform two negotiation sessions: we play with the agent \texttt{SimpleAgent} against itself, and we play Boulware vs Conceder and discuss the outcome. When we play SimpleAgent against itself, we notice the behaviour is quite random. This is explained by having a loo at the code; it always offers a random bid higher than a certain threshold,
by default this threshold is zero. Therefore it performs a \emph{random walk} through the 
bidding space. The acceptance strategy works as follows: it calculates a probability for each bid and the higher the bid, the higher the probability. If the deadline approaches it will accept all bids with higher probability. It will however never accept bids with a utility of zero. The equation that determines the probability of acceptance is:
\begin{align}
P = \frac{u - 2ut + 2(-1 + t + \sqrt{((-1 + t)^2 + u(-1 + 2t)})}{-1 + 2t}
\end{align}

Where $P$ denotes the probability of acceptance, $u$ denotes the utility of the bid and $t$ denotes the time in a range of $[0, 1]$. Because of this behaviour we found the average utility playing against itself is around $0.68$, and the time needed to reach an agreement was very short: $0.03$ (averaged over $20$ negotiations on the party domain).
This is quite logical, since both sides offer random bids, it is the
case quite soon that either bid has a high probability of acceptance for the opposing party. When this agent negotiates with other agents, it performs badly, because
it offers random bids. Smart agents will accept good offers and deny other offers,
and because of this the obtained utility for the simple agent will be quite low.
We confirm this by letting simple agent play against Boulware. In these tests the simpleagent obtains an average utility of $0.6$ while the Boulware agent scores on average $1.0$. 

\subsubsection{Performance of the \texttt{Boulware} vs. \texttt{Conceder}}
Finally, we play with the Boulware agent against the conceder agent. A typical negotiation trace is displayed below: \\

\begin{figure}[H]
\begin{center}
 \includegraphics[width=0.9\textwidth]{traceConcederBoulware.png}
 \caption{The \emph{Pareto Optimal Frontier} is displayed with a red line in this figure.}
 \label{fig:pareto} 
\end{center}
\end{figure}

In this figure, the green and blue lines are the bidding-traces for respectively the Boulware and the Conceder agent. The boulware agent chooses the best bid for itself, and keeps offering this bid. This agent only concedes at the very last moments before the deadline. The conceder agent on the other hand starts conceding directly and keeps on conceding every new bid. We ran simulations on the party domain using these agents. Averaging over 20 negotiations the Conceder reached an utility of $0.67$ and
Boulware $0.99$. 
 
\subsection{PEAS Description}

The first step in designing an agent is to specify the environment as fully as possible. In order to do this we specify the PEAS description which lists the following aspects of the environment: \emph{Performance}, \emph{Environment}, \emph{Actuators} and \emph{Sensors}. This measure is extensively discussed in Russel and Norvig \cite{russel-norvig}, so we adapt their notation.

\begin{table}[H]
  \small
    \begin{tabular}{|p{1.8cm}|p{3cm}|p{3cm}|p{3cm}|p{3cm}|}
    \hline
    \textbf{Agent} & \textbf{Performance \mbox{Measure}} & \textbf{Environment} & \textbf{Actuators} & \textbf{Sensors} \\
    \hline
    BOA Agent & Own $($versus opponents$)$ discounted final utility & Negotiation space defined by Genius, \mbox{Opponents} & New bid offering, \mbox{accepting/rejecting} offers & Biddings of the \mbox{opponent} agent \\
    \hline
    \end{tabular}
    
    \caption{PEAS description for our negotiation agent \label{table:peas-description}}
\end{table}

Table~\ref{table:peas-description} displays the PEAS description for our agent. We will briefly discuss each of the descriptions. First the \emph{environment} our agent operates in, this is the negotiation space as defined by Genius. It contains all the possible bids, each resulting in an individual utility for the agents. The \emph{performance} is measured by the utility for the agent, which in this project has to be as high as possible $($with a maximum of $1)$ and preferably equal to or higher than the opponents utility. There are two \emph{actuators}, first of all the acceptance or rejection of a bid of an opponent. The second actuator is the offering of a new bid. The bid is based on the biddings of the opponent, which is the only \emph{sensor} the agent has. Under the surface of the PEAS elements, different java-classes, algorithms and calculations are performed, like the \textit{nash-point} and the \textit{Kalai-Smorodinsky solution}. These will be explained in more detail in the next sections.

\subsection{BOA Framework}

Most negotiation agents consist of three components: \emph{Bidding strategy}, \emph{Opponent model} and \emph{Acceptance strategy}. Together these three components form the \emph{BOA framework}. As discussed by Baarslag et al., the advantages of separating the components using this framework are threefold \cite{baarslag2012decoupling}. In designing our negotiation agent the most useful advantage was the fact that it allows for easily changing individual components and analyze the interaction between the components. Below we provide a small description for the three required components.

\begin{description}
  \item[Bidding Strategy] \hfill \\
  The bidding strategy (BS) is arguably the most important component. It's goal is to determine the next bid that will be offered to the opponent. In determining the next bid the agent can interact with the opponent model to provide a suitable offer. Sometimes the negotiation session is split up in multiple phases and each of the phases use a different bidding strategy.

  \item[Opponent Model] \hfill \\
  Analyzing the opponents behaviour is essential for being able to determine the opponent's preference profile.  With a well defined opponent model (OM), the opponent's utility of a bid can be determined more accurate. In practice this works by estimating the opponents \emph{preference profile}, which is usually done by adapting \emph{Bayesian modeling} or \emph{frequency analysis}. 

  \item[Acceptance Strategy] \hfill \\
  Negotiation sessions ideally end by reaching an agreement. The acceptance strategy (AS) decides whether the opponents offer should be accepted or a new bid has to be offered. A broad variety of acceptance strategies is available with combinations of time-dependend, utility-dependent and threshold-dependent strategies.

\item[Opponent Model Strategy] \hfill \\
  Analyzing the opponents behaviour could, except for determining the opponent's preference profile, also be used to determine the strategy the opponent is using, like \emph{Hardheaded}, \emph{Conceder}, \emph{Tit-for-Tat}, etc. The own bidding strategy can be adapted to the opponent's model strategy (OMS), to derive a maximum possible outcome for the own utility.

\end{description}

\section{Strategy}
\label{sec:strategy}

\subsection{Bidding Strategy}
\label{sec:strategyBS}
Our bidding strategy is time dependent, we have implemented three phases each of which uses its own strategy for offering bids. The main \texttt{Group7\_BS} class chooses the strategy by evaluating the normalized time (\texttt{getTime()}) and the two hard-coded phase thresholds stored in the array \texttt{phaseBoundary}. In this section we discuss the strategy behind each of the phases separately.

\subsubsection{First Phase}
Our agents first bid is determined by the method \texttt{determineOpeningBid()}. By offering a bid of maximal utility the opponent can easily estimate our preference profile. Due to this reason we have chosen to offer an initial bid of utility $0.9$. The best bid of this utility is obtained by calling \texttt{getBidNearUtility()} of the \texttt{OutcomeSpace} class. \\

After determining our openings bid and receiving the initial bid from the opponent our agent continues by generating random bids. These random bids are generated within a small utility range. The (average) height of this range decreases by a small amount when time passes. The range is calculated by the method \texttt{getBidRange()} given in Listing~\ref{code:getrangefunctionfirstphase}. After calculating the range, the \texttt{getRandomBid()} method is used to generate a bid within this range. \\

We set the parameters such that in the beginning bids are generated around an utility of $1$, that is $[0.98, 1.0]$, and then gradually decreases towards the end. At the end of the phase the bids are centered around an utility of $0.9$ which corresponds to $[0.88, 0.92]$. As these numbers indicate, we set the margin of the bid range to $0.02$. \todo{Update this, we now also look at the best bid of the opponent.}

\clearpage

\begin{lstlisting}[caption=Code for calculating bid range as function of normalized time, label=code:getrangefunctionfirstphase]
public Range getBidRange (double t, double margin) {
  double normTime = t/this.phaseEnd; // Normalized time
  
  // Center of the utility range
  double val = 1-(normTime/10);
  
  // Best bid that the opponent has offered so far
  BidDetails bestOpponent = negotiationSession.getOpponentBidHistory().getBestBidDetails();
  
  // Set the bounds for the range
  double lb = val-margin;
  double ub = val+margin;
  
  if (bestOpponent.getMyUndiscountedUtil() > val){
    // The opponent has offered a better bid (for us)
    // than the center of our bid range, we counter this
    // by choosing a bid higher (+0.05) than the opponents bid.
    val = bestOpponent.getMyUndiscountedUtil()+0.05;
    
    // Update boundaries
    lb = val; ub = val+margin;
  }
  
  // Range in which bids are randomly generated
  Range r = new Range(lb, ub); 
  
  // Set upper bound to 1 if it exceeds upper bound
  if (r.getUpperbound() > 1) r.setUpperbound(1.0);
  
  return r;
}
\end{lstlisting}

\subsubsection{Second Phase}

\subsubsection{Third Phase}

\subsection{Opponent Model}
\label{sec:strategyOM}
Our opponent model is based on the standard frequency model, but has some tweaks that attempt to improve the quality of the model. The issue weight estimation is still equal to the standard frequency model method. The difference is in the way we approximate values of the items within an issue. The improvement is based on the idea that if the opponent does a new, previously unseen, bid, that this is probably close in utility to the other bids we've seen. This is implemented by changing the so-called learn value addition based on how ``wrong'' our opponent model is for the current bid. 

In order to estimate how ``wrong'' the estimate is, we first need to decide what offer we expect from the opponent. To do this we make the assumption that new offers the opponent makes are always close to old offers, but usually a bit lower. We also assume that the opponent's utility space has a uniformly distributed utility space. 

\subsection{Acceptance Strategy}
\label{sec:strategyAS}
% !TEX root = negotiation_final_report.tex
During the development of our acceptance strategy BOA component, $AS_{BOAconstructor}$, we designed, implemented, and tested many different acceptance conditions. Some failed horribly, but some exceeded our expectations. Within this paper we will only discuss the acceptance conditions that made it into the final product that was delivered along with this report.

Our acceptance strategy component consists of 7 acceptance conditions that are mainly based on the simple acceptance conditions as found in \cite{baarslag2013acceptance}. However, we modified these basic conditions according to our theories. To the best of our prior knowledge, they should all perform better together than any other acceptance strategy. Further along this report, we will discuss the tests we performed to evaluate our claims.

In the following subsections, we will discuss all 7 of our conditions included in the \texttt{determine- Acceptability()} method of our $AS_{BOAconstructor}$.

\begin{description}
  \item[AC Curve] \hfill \\
This acceptance condition is based on the simple $AC_{const}$\cite{baarslag2013acceptance}, but we added an extra dimension. $AC_{curve}$ depends on the normalized current time, $t_{current}$, of the negotiation and modifies the constant accordingly. We created a formula that would exponentially decrease when $t_{current}$ is close to $t_{end} = 1.0$ but would show almost non decreasing properties close to the beginning to the negotiation sessions. In the final version of $AC_{curve}$, we let it start at a utility boundary $f_{curve}(0.0) = 1.0$, and approach the first bid of the opponent, $U_{other}^{1}$, so that $f_{curve}(1.0) = (1 - \alpha) * 1.0 + \alpha * U_{other}^{1}$, where $\alpha$ is a percentage. This was preferred over approaching a static value, as it accommodates for varying preference profiles. However, we hereby assume that the opponent will most probably first send his best bid, in order to get his highest utility, namely $U_{opponent}^{1} = 1.0$. Until now we have not seen any agent that disagrees with this assumption, except the Simple Agent agent, which always selects a random bid. \\

The actual formula for the curve, implemented in the \texttt{getAcceptCurveValue()} method, is defined as $f_{curve}(t) = \mu * min(1.0, (1 + \lambda - t^2)^{\phi})$, where $\mu$ is the starting value of the curve, and $\lambda$ is a tiny value that allows the curve to end at a real value and at $f_{curve}(1.0)$ is not an asymptote. Furthermore, $\phi$ is the exponent which is calculated by solving for $f_{curve}(1.0) = (1 - \alpha) * 1.0 + \alpha * U_{other}^{1}$. The values used in our final version are $\alpha = 0.5$, $\mu = 1.0$, $\lambda = 0.005$.

  \item[AC Panic] \hfill \\
Our $AC_{panic}$ condition is based on $AC_{time}$\cite{baarslag2013acceptance}, but instead of looking at a threshold for a normalized time value, we approximate the amount of bids that are possibly left and only accept after less than $\beta$ bids are left. This is done by calculating the time, $\Delta~t_{bid}$, that was needed for each bid, with the help of a running average. The remaining time, $t_{remaining} = 1 - t_{now}$, is divided by the averaged time per bid to approximate the amount of bids that are left: $\#_{bids~left} = round(t_{remaining} / \Delta t_{bid})$. This is implemented in the \texttt{guessBidsLeft()} method of our $AS_{BOAconstructor}$. The value for $\beta$ used in our final version is $\beta = 3$.\\

Furthermore, two extra conditions were used before we accept with $AC_{panic}$. Firstly, if the last bid of the opponent was better than the perceived Kalai point minus a conceding value, say $\gamma = 0.05$, we accept for sure, since the received bid will be almost as good as a win-win or even better. Secondly, since we are in 'panic mode', we will also always accept when the opponent offers us a better offer than his best offer yet minus the same conceding value $\gamma$.

  \item[AC Horror] \hfill \\
This condition is also based on $AC_{time}$, just like our previous condition. However, $AC_{horror}$ will always accept whenever we think that only 1 bid is left for us to propose, $\#_{bids~left} = 1$, as it is always worse to end up with no agreement.

  \item[AC Kalai] \hfill \\
Whenever our utility of the opponent's last bid is really close to the Kalai point, $U_{kalai} - 0.01 <= U_{opponent}^{last} <= U_{kalai} + 0.01$, $AC_{kalai}$ always accepts, as this is a nice win-win situation. We only do this, though, when we are pretty sure that our calculated Kalai point is reliable. We do this by checking whether the opponent has made enough distinct bids to have an accurate enough opponent model.

  \item[AC Nash] \hfill \\
The acceptance condition $AC_{nash}$ is exactly the same as our $AC_{kalai}$, except for the use of the Nash point instead of the Kalai point.

  \item[AC Worst] \hfill \\
This condition is inspired on $AC_{next}$\cite{baarslag2013acceptance}, but instead of accepting when the received bid is higher than our next bid, we accept when the opponent's last bid is higher than our worst bid, minus an time-based increasing conceding value $\delta(t) = t * 0.05$, so that we accept when $U_{opponent}^{last} > U_{our}^{worst} - \delta(t_{now})$. However, to prevent this value from accepting too quickly, we created a time-dependent formula that we use to cut off the said threshold. The formula is defined as follows, $threshold(t) = max(\epsilon * (1-time) + \zeta, U_{our}^{worst} - \delta(t_{now}))$, where in our final version $\epsilon = 0.3$ and $\zeta = 0.6$. This way, our $AC_{worst}$ will never accept anything below a utility of 0.6, and when $t_{now} = 0.0$, the condition will never accept anything below 0.9.

  \item[AC Next] \hfill \\
The last acceptance condition is simply the $AC_{next}$ as mentioned in \cite{baarslag2013acceptance}.

\end{description}

Finally, our acceptance strategy only looks back within a sliding time window of $\omega = 0.2$, so that the behavior of most of the above mentioned acceptance conditions varies during the course of a negotiation session. This is to accommodate for changing behaviors in our and the opponent's strategies. Thus, we never look at the entire time line of a negotiation session, in order to determine the acceptability of the opponent's last bid.

\subsection{Opponent Model Strategy} 
\label{sec:strategyOMS}

While an opponent strategy model is not required for a good negotiation agent we have decided to implement a simple version of this component. Our strategy model has one method \texttt{getOpponentModel()} that approximates the behaviour of the opponent. \\

While testing our agent we noticed that \emph{hardheaded} opponents do not offer a large number of different bids, i.e. they stick to bids that have high utility for themselves. On the other hand, agents that act as \emph{conceder} offer a wide variety of bids due to their conceding behaviour. Our strategy model uses this information to determine the strategy of the opponent. \\

Our strategy model first fetches the bid history of the oponent, then these results are filtered on time. Only opponent bids that were offered before $t=0.6$ are taken into account. Then our helper function \texttt{getDistinctBids()} returnes a list of unique bids, i.e. all different bids are returned only once. The number of distinct bids is compared to the total number of opponent bids by calculating the ratio between the two. This ratio is compared to a hardcoded threshold; empirical measurements suggested a threshold of $0.03$ for best results. A ratio below this threshold suggests the opponent to be \emph{hardheaded} since the number of unique bids is small. The method \texttt{getOpponentModel()} then returns that the opponent is hardheaded, while it returns \emph{conceder} otherwise. \\

For our agent, the classification (\emph{hardheaded} vs \emph{conceder}) of the opponent is sufficient. \todo{This section is not yet finished since it is unclear where we are using the OMS...} \\

This is a very simple though efficient method for determining the opponents strategy. For future work we also suggest calculating the \emph{variance} between the opponents bids. A small variance would suggest the agent sticks to its own best utility and can be classified as hardheaded. This approach looks promising, however we did not have enough time to implement is.


\section{Testing \& Performance}
\label{sec:performance}
After designing and implementing our agent the BOAconstructor it is time to put it to the test in various situations. In the next two subsections we quantify the performance of our agent. We present the results of a wide range of tests and discuss the effeciency of the agent in terms of reaching \emph{Nash solutions} and outcomes that lie on the \emph{Pareto Optimal Frontier} (POF).

\subsection{Performance against \texttt{ANAC 2011} agents}

In this part we compare our agent to three of the agents that attended in the ANAC 2011 competitio: \emph{HardHeaded}, \emph{Gahboninho} and \emph{The Negotiatior}. In addition to those agents we also negotiate with ourself. The results have been obtained by running a tournament using multiple utility profiles for the Party domain and setting the negotiation time to $30$ seconds. \\

\begin{table}
	\centering
	\small
    \begin{tabular}{l|p{2cm}|p{2cm}|p{2cm}|p{2cm}|p{2cm}|p{2cm}|}
    ~              & Average Time Agreement & Average Discounted Util & Average Dist. to Nash & Average Dist. to Pareto & Average Dist. to Kalai \\
    \hline
    \emph{HardHeaded}		& 0.858  & 0.927  & 0.102  & 0.017  & 0.084   \\ \hline
    \emph{Gahboninho}   	& 0.887  & 0.918  & 0.032  & 0.001  & 0.035   \\ \hline
    \emph{The Negotiator} 	& 0.725  & 0.871  & 0.050  & 0.003  & 0.052   \\ \hline
    \emph{Self}             & 0.823  & 0.877  & 0.061  & 0.007  & 0.057   \\ \hline
    \end{tabular}
    \caption{Performance of our agent against \texttt{ANAC 2011} agents (Runtime: $30$s) \label{table:anac2011-results}}
\end{table}

From the results in Table~\ref{table:anac2011-results} we notice a fairly high average discounted utility against the other agents. Regarding the effiency of the negotiation we observe that the average distance to the \emph{Pareto Optimal Frontier} upon agreement is very small. When playing against \texttt{The Negotiator} we see than on average the agreement lies on the frontier.

\todo{Check these conclusions!}

\subsection{Choosing Different Acceptance Strategies}

The next test consists of running our agent with varying \emph{acceptance strategies}. We perform those test so we are able to compare our acceptance strategy with other available ones (our strategy is discussed in Section~\ref{sec:strategyAS}). Again these tests are performed by starting a small tournament. \\

\todo{Insert results and discuss them...}

\subsection{Comparison with HardHeaded Frequency Model}
To test if our opponent model is better than the default hardheaded frequency model, we ran the performance test against the ANAC 2011 agents using the hardheaded frequency model. The results of this are shown in \autoref{table:anac2011-hh}.
\begin{table}[H]
	\centering
	\small
    \begin{tabular}{l|p{2cm}|p{2cm}|p{2cm}|p{2cm}|p{2cm}|p{2cm}|}
    ~              & Average Time Agreement & Average Discounted Util & Average Dist. to Nash & Average Dist. to Pareto & Average Dist. to Kalai \\
    \hline
    \emph{HardHeaded}		& 0.883  & 0.916  & 0.166  & 0.055  & 0.152   \\ \hline
    \emph{Gahboninho}   	& 0.881  & 0.848  & 0.116  & 0.020  & 0.097   \\ \hline
    \emph{The Negotiator} 	& 0.756  & 0.868  & 0.092  & 0.027  & 0.079   \\ \hline
    \emph{Self}             & 0.839  & 0.806  & 0.125  & 0.035  & 0.140   \\ \hline
    \end{tabular}
    \caption{Performance of our agent with \texttt{HardHeaded Frequency Modeling} (Runtime: $30$s) \label{table:anac2011-hh}}
\end{table}
We see that, compared to our own opponent model, the average utility has decreased by $0.071$, which means that our opponent model leads to  better results for our agents. 
Furthermore, we see that the average utilities for the opponent are also lower compared to what our opponent model achieved.
The distance to the \emph{Nash point}, \emph{Pareto Frontier} and \emph{Kalai point} are especially interesting for evaluating the quality of the opponent model: a better opponent model allows us to offer bids closer to these optimal values. We also note here that for every agent these properties are lowest using our own opponent model.

\subsection{Performance In Different Domains}
In order to test whether our agent can cope with different domains, we tested it against the same selection of 2011 agents, but using different domains. More specifically, the test was run in the Car/ADG domain, the Amsterdam party domain, the Camera domain, the Nice Or Die domain, the Grocery domain, the IS BT Acquisition IS prof domain, and the Laptop domain. The average results for all agents are shown in \autoref{table:anac2011-domains}.
\begin{table}[H]
  \centering
  \small
  \begin{tabular}{l|p{2cm}|p{2cm}|p{2cm}|p{2cm}|p{2cm}|}
    ~                     & Average Time Agreement & Average Discounted Util & Average Dist. to Nash & Average Dist. to Pareto & Average Dist. to Kalai \\
    \hline
    \emph{HardHeaded}     & 0.807 & 0.750 & 0.142 & 0.032 & 0.150 \\ \hline
    \emph{Gahboninho}     & 0.731 & 0.807 & 0.065 & 0.004 & 0.165 \\ \hline
    \emph{The Negotiator} & 0.694 & 0.646 & 0.288 & 0.000 & 0.220 \\ \hline
    \emph{Self}           & 0.744 & 0.712 & 0.165 & 0.012 & 0.179 \\ \hline
  \end{tabular}
 \caption{Performance of our agent using a set of diverse domains (Runtime: $30$s) \label{table:anac2011-domains}}
\end{table}



\subsection{How Generic is our Agent?}

In the previous tests we only observed the behaviour of the agent on the party domain. In order to test how generic our agent is we also perform a number of tests on other scenarios. In addition to different domains we also incorporate \emph{discounts} for the utility. \\
\todo{Insert test results and discuss results...}





\section{Conclusion}
\label{sec:conclusion}
\todo{a conclusion in which you summarize your experience as a team with regards to building the negotiating agent and discuss what extensions are required to use your agent in real-life negotiations to support (or even take over) negotiations performed by humans.} \\ 

\todo{Het lijkt me hier op zijn plaats om het BOA framework te bekritiseren zoals het nu is. Dat je sommige componenten niet makkelijk kan bereiden (BS/AS) is echt heel onhandig...}



\bibliography{negotiation_final_report}
\bibliographystyle{plain}

\end{document}

\documentclass[a4paper,10pt]{article}
%\documentclass[a4paper,10pt]{scrartcl}

\usepackage[utf8]{inputenc}
\usepackage{amsmath}
\usepackage{listings}
\usepackage{hyperref}
\usepackage{catoptions}
\usepackage[margin=1in]{geometry}
\usepackage{color}
\usepackage{soul}
\usepackage{float}
\usepackage{framed}
\usepackage[sc]{mathpazo}
\linespread{1.20}         % Palatino needs more leading (space between lines)
\usepackage[T1]{fontenc}
\usepackage{microtype}
\usepackage{enumerate}
\usepackage{courier}
\usepackage{graphicx}

\newcommand{\authorbd}{Bas Dado (4033736)}
\newcommand{\authoroh}{Olivier Hokke (1352679)}
\newcommand{\authortr}{Tom Runia (1517996)}
\newcommand{\authoras}{Arnold Schutter (4260724)}
\newcommand{\authortv}{Tom Viering (4333055)}
\newcommand{\maintitle}{IN4010: Negotiation Project}
\newcommand{\subtitle}{Group 7 - BOAconstructor}

\title{\maintitle\\\subtitle}
\author{\authorbd\\\authoroh\\\authortr\\\authoras\\\authortv}
\date{\today}

\pdfinfo{%
  /Title    (\maintitle - \subtitle)
  /Author   (\authorbd, \authoroh, \authortr, \authoras, \authortv)
  /Creator  (\authorbd, \authoroh, \authortr, \authoras, \authortv)
  /Producer (\authorbd, \authoroh, \authortr, \authoras, \authortv)
  /Subject  (Automated Negotiation)
  /Keywords (Automated Negotiation, Genius, Bidding Strategy, Acceptance Strategy, Opponent Model, Opponent Model Strategy)
}

% Settings for hyperref package (e.g. wat \autoref en \nameref moeten doen)
\hypersetup{
  colorlinks  = true,
  linkcolor   = [rgb]{0.1,0.1,0.5},
  citecolor   = [rgb]{0.5,0.1,0.1},
  filecolor   = [rgb]{0.1,0.5,0.5},
  urlcolor    = [rgb]{0.1,0.1,0.7}
}

% Adds the command "\Autoref" to make it possible to use a capital in the referenced object name
\makeatletter
\def\figureautorefname{figure}
\def\tableautorefname{table}
\def\Autoref#1{%
  \begingroup
  \edef\reserved@a{\cpttrimspaces{#1}}%
  \ifcsndefTF{r@#1}{%
    \xaftercsname{\expandafter\testreftype\@fourthoffive}
      {r@\reserved@a}.\\{#1}%
  }{%
    \ref{#1}%
  }%
  \endgroup
}
\def\testreftype#1.#2\\#3{%
  \ifcsndefTF{#1autorefname}{%
    \def\reserved@a##1##2\@nil{%
      \uppercase{\def\ref@name{##1}}%
      \csn@edef{#1autorefname}{\ref@name##2}%
      \autoref{#3}%
    }%
    \reserved@a#1\@nil
  }{%
    \autoref{#3}%
  }%
}
\makeatother

% Settings for listings of java code
\definecolor{mygreen}{rgb}{0,0.6,0}
\definecolor{light-gray}{gray}{0.95}
\lstset{basicstyle=\footnotesize\ttfamily,breaklines=true,language=Java}
\lstset{frame=single,commentstyle=\color{mygreen},keywordstyle=\color{blue}}
\lstset{aboveskip=0.5cm,belowskip=0.3cm}
\lstset{backgroundcolor=\color{light-gray}}

% Define the todo command
\newcommand{\todo}[1] {\hl{TODO: #1}}
\setlength{\parindent}{0cm}

\begin{document}
\maketitle

\section{Introduction}
\label{sec:introduction}
A relatively new and evolving branch of Artificial Intelligence is automated negotiation. In automated negotiation, two or more agents negotiate about a multi-issue problem in order to find a solution that maximizes their utility. The utilities for each possible bid are determined using a human-defined preference profile, which consists of weights for each of the issues and a utility for each possible value of the issues. 

In this report we describe the process of creating an agent for automated negotiation. \Autoref{sec:exercises} contains the answers to the questions posed in the assignment concerning the party domain and genius in general. 
\Autoref{sec:strategy} describes the strategy our agent uses. The main chapter contains the high-level description of the agent. In it's subsections, \autoref{sec:strategyAS}, \autoref{sec:strategyBS}, \autoref{sec:strategyOM} and \autoref{sec:strategyOMS}, we go into more detail about the specific BOA components. \Autoref{sec:performance} shows the results of performance tests of our final agent against some other agents that were included in genius. In \Autoref{sec:conclusion} we describe our experience regarding building the agent and decide what is needed in order to use our agent in real world negotiations.

\newpage
\tableofcontents
\newpage


\section{Exercises}
\label{sec:exercises}

\subsection{Party domain analysis}

\subsubsection{Bidding space and pareto frontier}

In this subsection we analyse the party domain and compute the Pareto frontier.
The party domain has 6 issues: food (4 options), drinks (4 options), 
locations (4 options), invitations (4 options), music (3 options), cleanup (4 options).
This results in $4 \times 4 \times 4 \times 4 \times 3 \times 4 = 3072$ outcomes.
We took the preference profile of user 10 and 15 of the party domain,
we compute the outcome space and the pareto frontier and plot this in the figure below:

%\begin{figure}[ht]
\begin{center}
 \includegraphics[width=0.6\textwidth]{pareto.png}
% \caption{The pareto optimal frontier is displayed with a red line in this figure.}
% \label{fig:pareto} 
\end{center}
%\end{figure}

\subsubsection{Analyse simple opponents}

Finally, we perform two negotiation sessions: we play with the agent SimpleAgent against
itself, and we play Boulware vs Conceder and explain the outcome. 

When we play simple agent against itself, we notice the behaviour is quite random.
This is explained by the code: it always offers a random bid higher than a certain threshold,
by default this threshold is zero. Therefore it performs a random walk through the 
bidding space. It's acceptance strategy is as follows: it calculates a probability for each bid, the higher the bid, the higher the probability. If the deadline is getting closer it will accept all bids with higher probability. It will however never accept bids with a utility of zero. The equation that determines the 
probability of acceptance is:

\begin{equation}
P = \frac{u - 2ut + 2(-1 + t + \sqrt{((-1 + t)^2 + u(-1 + 2t)})}{-1 + 2t}
\end{equation}

Where $P$ denotes the chance of acceptance, $u$ denotes the utility of the bid and $t$ denotes the time in a range of $[0,1]$.

Because of this behaviour we found the average utility playing against itself is around $0.68$, and the
time needed to reach an agreement was very short: $0.03$ (averaged
over 20 negotiations on the party domain).
This is quite logical, since both sides offer random bids, it is the
case quite soon that either bid has a high probability of acceptance for the opposing party.
When this agent negotiates with other agents, it performs badly, because
it offers random bids. Smart agents will accept good offers and deny other offers,
and because of this the obtained utility for the simple agent will be quite low.
We confirm this by letting simple agent play against boulware. In these tests the simple agent obtains 
an average utility of $0.6$ while the boulware agent scores on average $1.0$.

Finally, we play with the boulware agent against the conceder agent. A typical negotiation trace is displayed below:

%\begin{figure}[ht]
\begin{center}
 \includegraphics[width=0.6\textwidth]{traceConcederBoulware.png}
% \caption{The pareto optimal frontier is displayed with a red line in this figure.}
% \label{fig:pareto} 
\end{center}
%\end{figure}
In the figure, the green and blue lines are the bidding-traces for respectively the boulware and the conceder agent. The boulware agent chooses the best bid for itself, and keeps offering this bid. This agent
only concedes at the very last moments before the deadline. The conceder agent on the other hand
starts conceding directly and keeps on conceding every new bid. We ran simulations on the party domain using these agents. Averaging over 20 negotiations the conceder reached an utility of $0.67$ and
boulware $0.99$. 
 
\subsection{PEAS Description}

The first step in designing an agent is to specify the environment as fully as possible. In order to do this we specify the PEAS description which lists the following aspects of the environment: \emph{Performance}, \emph{Environment}, \emph{Actuators} and \emph{Sensors}. This measure is extensively discussed in Russel and Norvig \cite{russel-norvig}, so we adapt their notation.

\begin{table}[H]
    \begin{tabular}{|p{1.8cm}|p{3cm}|p{3cm}|p{3cm}|p{3cm}|}
    \hline
    \textbf{Agent} & \textbf{Performance \mbox{Measure}} & \textbf{Environment} & \textbf{Actuators} & \textbf{Sensors} \\
    \hline
    BOA Agent & Own $($versus \mbox{opponents}$)$ discounted final utility & Negotiation space defined by Genius, \mbox{Opponents} & New bid offering, \mbox{Accepting/rejecting} offers & Java classes that return the bidding history of the \mbox{opponent} agent \\
    \hline
    \end{tabular}
    
    \caption{PEAS description for our negotiation agent \label{table:peas-description}}
\end{table}

Table~\ref{table:peas-description} displays the PEAS description for our agent. We will briefly discuss each of the descriptions. First the \emph{environment} our agent operates in, this is the negotiation space as defined by Genius. \todo{Check the PEAS description I came up with and finish the additional description} 

\subsection{BOA Framework}

Most negotiation agents consist of three components: \emph{Bidding strategy}, \emph{Opponent model}, \emph{Acceptance strategy}. Together these three components form the \emph{BOA framework}. As discussed by Baarslag et al., the advantages of separating the components using this framework are threefold \cite{baarslag2012decoupling}. In designing our negotiation agent the most useful advantage was the fact that it allows for easily changing individual components and analyze the interaction between the components. Below we provide a small description for the three required components.

\begin{description}
  \item[Bidding Strategy] \hfill \\
  The bidding strategy is arguably the most important component. Its goal is to determine the next bid that is offered to the opponent. In determining the next bid the agent can interact with the opponent model to provide a suitable offer. Sometimes the negotiation sessions is split up in multiple phases and each of the phases uses a different bidding strategy.

  \item[Opponent Model] \hfill \\
  Learning the opponents behaviour is essential in a good negotiation session. It allows for adapting to the way the other party provides its bids. In practice this works by estimating the opponents \emph{preference profile}, which is usually done by adapting \emph{Bayesian modelling} or \emph{frequency analysis}.

  \item[Acceptance Strategy] \hfill \\
  Negotiation sessions ideally end by reaching an agreement. The acceptance strategy decides whether the opponents offer should be accepted. A broad variety of acceptance strategies is available reaching from postponing acceptance until the last step and aiming at fast decisions.

\end{description}

\section{Strategy}
\label{sec:strategy}

\subsection{Bidding Strategy}
\label{sec:strategyBS}
Our bidding strategy is time dependent, we have implemented three phases each of which uses its own strategy for offering bids. The main \texttt{Group7\_BS} class chooses the strategy by evaluating the normalized time (\texttt{getTime()}) and the two hard-coded phase thresholds stored in the array \texttt{phaseBoundary}. In this section we discuss the strategy behind each of the phases separately.

\subsubsection{First Phase}
Our agents first bid is determined by the method \texttt{determineOpeningBid()}. By offering a bid of maximal utility the opponent can easily estimate our preference profile. Due to this reason we have chosen to offer an initial bid of utility $0.9$. The best bid of this utility is obtained by calling \texttt{getBidNearUtility()} of the \texttt{OutcomeSpace} class. \\

After determining our openings bid and receiving the initial bid from the opponent our agent continues by generating random bids. These random bids are generated within a small utility range. The (average) height of this range decreases by a small amount when time passes. The range is calculated by the method \texttt{getBidRange()} given in Listing~\ref{code:getrangefunctionfirstphase}. After calculating the range, the \texttt{getRandomBid()} method is used to generate a bid within this range. \\

We set the parameters such that in the beginning bids are generated around an utility of $1$, that is $[0.98, 1.0]$, and then gradually decreases towards the end. At the end of the phase the bids are centered around an utility of $0.9$ which corresponds to $[0.88, 0.92]$. As these numbers indicate, we set the margin of the bid range to $0.02$. \todo{Update this, we now also look at the best bid of the opponent.}

\clearpage

\begin{lstlisting}[caption=Code for calculating bid range as function of normalized time, label=code:getrangefunctionfirstphase]
public Range getBidRange (double t, double margin) {
  double normTime = t/this.phaseEnd; // Normalized time
  
  // Center of the utility range
  double val = 1-(normTime/10);
  
  // Best bid that the opponent has offered so far
  BidDetails bestOpponent = negotiationSession.getOpponentBidHistory().getBestBidDetails();
  
  // Set the bounds for the range
  double lb = val-margin;
  double ub = val+margin;
  
  if (bestOpponent.getMyUndiscountedUtil() > val){
    // The opponent has offered a better bid (for us)
    // than the center of our bid range, we counter this
    // by choosing a bid higher (+0.05) than the opponents bid.
    val = bestOpponent.getMyUndiscountedUtil()+0.05;
    
    // Update boundaries
    lb = val; ub = val+margin;
  }
  
  // Range in which bids are randomly generated
  Range r = new Range(lb, ub); 
  
  // Set upper bound to 1 if it exceeds upper bound
  if (r.getUpperbound() > 1) r.setUpperbound(1.0);
  
  return r;
}
\end{lstlisting}

\subsubsection{Second Phase}

\subsubsection{Third Phase}



\subsection{Opponent Model}
\label{sec:strategyOM}

\subsection{Acceptance Strategy}
\label{sec:strategyAS}

\subsection{Opponent Strategy Model}
\label{sec:strategyOMS}

\section{Testing \& Performance}
\label{sec:performance}
\todo{a section documenting the tests you performed to improve the negotiation strength of your agent. You must include scores of various tests over multiple sessions that you performed while testing your agent. Describe how you set up the testing situation and how you used the results to modify your agent.}


\section{Conclusion}
\label{sec:conclusion}
\todo{a conclusion in which you summarize your experience as a team with regards to building the negotiating agent and discuss what extensions are required to use your agent in real-life negotiations to support (or even take over) negotiations performed by humans.} \\ 

\todo{Het lijkt me hier op zijn plaats om het BOA framework te bekritiseren zoals het nu is. Dat je sommige componenten niet makkelijk kan bereiden (BS/AS) is echt heel onhandig...}


\bibliography{negotiation_final_report}
\bibliographystyle{plain}

\end{document}

The bidding strategy we chose for is not a static strategy but is time dependend. The strategy is divided in three different timephases with each phase using its own strategy for offering bids. The main \texttt{Group7\_BS} class decides the current phase by evaluating the normalized time (\texttt{getTime()}) and the two hard-coded phase thresholds, which are stored in the array \texttt{phaseBoundary}. In this section we discuss the strategy behind each of the phases separately.

\subsubsection{First Phase}
The agents first bid is determined by the method \texttt{determineOpeningBid()}. By offering a bid of maximal utility the opponent can easily estimate our preference profile. Due to this reason we have chosen to offer an initial bid of utility $0.9$. The best bid of this utility is obtained by calling \texttt{getBidNearUtility()} of the \texttt{OutcomeSpace} class. \\

After determining our opening bid and receiving the initial bid from the opponent, our agent continues by generating random bids. These random bids are generated within a small own-utility range. The (average) upperbound of this range decreases by a small amount when time passes. The range is calculated by the method \texttt{getBidRange()}. After calculating the range, the \texttt{getRandomBid()} method is used to generate a bid within this range. \\

We set the parameters such that in the beginning of the negotiation bids are generated around an utility of $1$ (in practice this is $[0.98, 1.0]$) and then gradually decreases towards the end. At the end of the phase the bids are centered around an utility of $0.9$ ( in practice this is $[0.88, 0.92]$). As these numbers indicate, we set the margin of the bid range to $0.02$. 

\todo{Update this, we now also look at the best bid of the opponent.}

\subsubsection{Second Phase}

For phase $2$ we decided to implement the \emph{Tit-for-Tat} strategy \cite{titfortat}.
We chose this because it works fairly well against different agents, since you will play
hard against hardheaded opponents and more nice against conceding agents.
However, as noted in the referenced paper, we should not play too nice,
and try to exploit other opponents. We use the following strategy. \\

We use the distance to the \emph{Kalai-Smorodinsky} point as reference to calculate
the amount of concession for both the opponent as our own agent. When conceding, we aim for this point.
We decided to use the distance to this point instead of calculating the distance for our own utility of opponent bids, since when we look at the latter, we can hardly see whether or not
our opponent is conceding (since these utilities will have large variation). So instead, we use the distance to the KS and this improves the detection of concessions. \\

We average the distance of the last five bids to the Kalai-Smorodinsky point to determine the concession of the opponent. We
multiply this by a factor $\sfrac{1}{3}$, and then match their concession to the KS point.
We do this to avoid conceding too much in the direction of the opponent. \\

Furthermore, to ensure our agent does not get stuck when playing against 
another Tit-for-Tat agent, we will randomly concede from time to time. 
These concessions are done towards the Kalai-Smorodinsky point, and linearly over 
$10$ bids. When the concession has been done, we hope the opponent matches 
our concession. If the opponent has done this, we will approach him
using Tit-for-Tat, but if not, we take our concession back. \\

Finally, we want to reach the most efficient outcomes. Because of this,
we will randomly offer bids on the \emph{Pareto Frontier} $50\%$ of the time.
However, to select this bid, we take the Pareto bid closest to our current bid.
When determining the closest bid, we weigh our utility twice as much
as the opponent utility, to ensure we don't concede too much by accident.

\subsubsection{Third Phase}

The third phase only consists of the very last part of the negotiation session (usually from $t=0.95$). The strategy that is employed here relies on the opponent model strategy. Our method \texttt{getOpponentModel()} from the OMS returns whether the opponent is assumed to be \emph{HardHeaded} or \emph{Conceder}. Based on what is returned by the OMS the next bid is determined as described below.

\begin{description}
  \item[Opponent is assumed to be Conceder] \hfill \\
  In this case the next bids that are offered will all be at the Kalai Point.

  \item[Opponent is assumed to be HardHeaded] \hfill \\
  Analyzing the opponents behaviour is...\todo{Fix after Olli changed his things...}
\end{description}

